%\documentclass[a4paper,12pt]{article}%report
 
 \documentclass[a4paper,12pt]{article}%report
 % das hier ist eine font
 %\usepackage{newcent}
 \usepackage{geometry}
 \geometry{
 	a4paper,
 	total={140mm,230mm},
 	left=35mm,
 	top=40mm,
 }
\usepackage{amsthm}
\usepackage[ngerman]{babel}     
\usepackage[utf8]{inputenc}      
\usepackage {amsmath}
\usepackage{bbm}
\usepackage{graphicx}
\usepackage{color}
\usepackage{hyperref}
\usepackage{tabularx}
\usepackage{booktabs}% http://ctan.org/pkg/booktabs
\newcommand{\tabitem}{~~\llap{\textbullet}~~}
\usepackage[arrow, matrix, curve]{xy}

\usepackage{multirow}
 \usepackage[table,xcdraw]{xcolor}
\usepackage{textpos}

%\newenvironment{ex}{begdef}{enddef}
% Theorem und Satz           
\newtheorem{thm}{Theorem}[section]
\newtheorem{myDef}[thm]{Definition}
% \newtheorem{thm}{Satz}[section]
\newtheorem{ex}{Aufgabe}[section]
% \newtheorem{myDef}[thm]{Definition}
\newtheorem{myExmpl}[thm]{Beispiel}
\newtheorem{myCmmt}[thm]{Bemerkung}
\newtheorem{lem}[thm]{Lemma}
\newtheorem{prop}[thm]{Proposition}
\newtheorem{myKor}[thm]{Korollar}
\newtheorem{myV}{Veranschaulichung}
\newtheorem{myLemma}[thm]{Lemma}

\usepackage[table,xcdraw]{xcolor}

\newcommand{\beg}[1]{\textsc{#1}}
\newcommand{\openEx}[1]{ {\bf Aufgabe #1:}}
\newcommand{\openAns}{ {\bf Antwort:~}}

% neue eigene shortcuts
\newcommand{\N}{\mathbbm N}
\newcommand{\K}{\mathbbm K}
\newcommand{\R}{\mathbbm R}
\newcommand{\A}{\mathcal{A}}
\newcommand{\B}{\mathcal{B}}
\newcommand{\F}{\mathcal{F}}
\newcommand{\M}{\mathcal{M}}
\newcommand{\HH}{\mathcal{H}}
\newcommand{\U}{\mathcal{U}}
\newcommand{\PP}{\mathbbm P}
%\newcommand{\r}{\mathbbm R}
\newcommand{\C}{\mathbbm C}
\newcommand{\Z}{\mathbbm Z}
\newcommand{\zuZeigen}{ \mathrm{Z\kern-.5em\raise-0.4ex\hbox{Z}}}
\newcommand{\hnot}[1]{{\it #1}}
\newcommand{\Hnot}[1]{{\bf #1}}

\newcommand{\refFootnote}[1]{\footnote{#1}}
\definecolor{rahmen}{rgb}{.7,0.9,0.8}    
\definecolor{grund}{gray}{.99}           
\definecolor{schrift_box}{cmyk}{0.9,0.7,0.8,0}   
\definecolor{whites}{cmyk}{0, 0, 0,0}   
\definecolor{schrift_normal}{gray}{0}   
\definecolor{ekelig}{cmyk}{0.1,0.9,0.9,0}   
\definecolor{ekeligHintergrund}{cmyk}{0.1,0.1,0.9,0}   
  
  \newcommand{\defWord}[1]{{\rm #1}}
\newcommand{\openBox}[1]{
   \color{grund}
  
  \fcolorbox{rahmen}{grund}{\fbox{ \color{schrift_box}\parbox{\dimexpr \linewidth - 2\fboxrule - 2\fboxsep}{#1}}}
  \color{schrift_normal}}
  
\newcommand{\marker}[1]{
   \color{ekelig}
\fcolorbox{whites}{ekeligHintergrund}{
    TODO: #1
}
    \color{schrift_normal}}
  
\newcommand{\openBoxx}[2]{
   \color{grund}
  
  \fcolorbox{rahmen}{grund}{\fbox{ \color{schrift_box}\parbox{\dimexpr \linewidth - 2\fboxrule - 2\fboxsep}{#1}}{#2}}
  \color{schrift_normal}}
  
% Das hier ist nur so einnkommentiert, da ich auf OS X nicht herausgefunden 
% habe, wie man mathbbm zum Laufen kriegt.
%\newcommand{\N}{N}
%\newcommand{\K}{K}
%\newcommand{\R}{R}%
%\newcommand{\C}{C}
%\newcommand{\Z}{Z}
%\newcommand{\zuZeigen}{Z}

\newcommand*{\nametag}[1]{%
  \stepcounter{equation}%
  \tag{\tagtext{#1}\tagcomma\tagnumber{\theequation}}%
}
\makeatletter% <-- anklicken liefert Erklärung
\newcommand*{\eqnumref}[1]{%
  \begingroup
    \let\tagtext\@gobble
    \let\tagcomma\relax
    \eqref{#1}%
  \endgroup
}
\newcommand*{\eqtextref}[1]{%
  \begingroup
    \let\tagcomma\relax
    \let\tagnumber\@gobble
    \ref{#1}%
  \endgroup
} 


\newcommand{\grafik}[2]{\begin{figure}[!htb]
		\noindent\includegraphics[width=\linewidth,height=\textheight,
		keepaspectratio]{#1}
		\caption{\textrm{#2}}%
	\end{figure}}
\usepackage{cite}


\newcommand{\grafikM}[2]{\begin{figure}[!htb]
		\noindent\includegraphics[width=12.5cm,height=6.5cm]{#1}
		\caption{\textrm{#2}}%
	\end{figure}}
	\usepackage{cite}
	

\newcommand{\grafikMNon}[2]{\begin{figure}[!htb]
		\noindent\includegraphics[width=\linewidth]{#1}
		\caption{\textrm{#2}}%
	\end{figure}}
	\usepackage{cite}

\newcommand{\grafikMS}[2]{\begin{figure}[!htb]
		\noindent\includegraphics[height=4cm]{#1}
		\caption{\textrm{#2}}%
	\end{figure}}
	\usepackage{cite}
	

\newcommand{\grafikMHalf}[2]{\begin{figure}[!htb]
		\noindent\includegraphics[width=12.5cm,height=2.5cm]{#1}
		\caption{\textrm{#2}}%
	\end{figure}}

\usepackage{cite}




\newcommand{\newDef}[1]{ {\bf #1} }

\newcommand{\oldDef}[1]{ {\it #1} }




\usepackage{blindtext}

\usepackage{fancyhdr}%Kopf- und Fußzeile
\renewcommand{\headrulewidth}{1pt} %Linie oben
\fancyhf{}
\fancyhead[L]{\leftmark} %Kopfzeile links bzw. innen
\fancyhead[R]{\thepage} %Kopfzeile rechts bzw. außen


%Das hier ist neu dazugekommen für automatische Wörtertrennung.
\usepackage{xspace}



\setlength\parindent{0pt}

\newif\ifcomment



% zum abaendern wenn die Erklaerungen ausgeblendet werden sollen
\newcommand{\invisible}[1]{#1}


\newcommand{\lecture}[2]{\color{black} 
	
	\begin{textblock*}{3cm}(-2.4cm,0.125cm)
		{\bf Lecture {#1}}
	\end{textblock*}
	
	\begin{textblock*}{3cm}(-2.4cm,0.625cm)
		#2
	\end{textblock*}
}
\newcommand{\tutorial}[2]{\color{black} 
	
	\begin{textblock*}{3cm}(-2.4cm,0.125cm)
		{\bf Tutorial {#1}}
	\end{textblock*}
	
	\begin{textblock*}{3cm}(-2.4cm,0.625cm)
		#2
	\end{textblock*}
}



%cpp stuff

%\setmonofont{Consolas} %to be used with XeLaTeX or LuaLaTeX
\definecolor{bluekeywords}{rgb}{0,0,1}
\definecolor{greencomments}{rgb}{0,0.5,0}
\definecolor{redstrings}{rgb}{0.64,0.08,0.08}
\definecolor{xmlcomments}{rgb}{0.5,0.5,0.5}
\definecolor{types}{rgb}{0.17,0.57,0.68}

\usepackage{listings}
\lstset{language=C++,
	captionpos=b,
	%numbers=left, %Nummerierung
	%numberstyle=\tiny, % kleine Zeilennummern
	frame=lines, % Oberhalb und unterhalb des Listings ist eine Linie
	showspaces=false,
	showtabs=false,
	breaklines=true,
	showstringspaces=false,
	breakatwhitespace=true,
	escapeinside={(*@}{@*)},
	commentstyle=\color{greencomments},
	morekeywords={partial, var, value, get, set},
	keywordstyle=\color{bluekeywords},
	stringstyle=\color{redstrings},
	basicstyle=\ttfamily\small,
}


% \newcommand{\erklaerung}[1]{}
%opening
\title{Advanced Enterprise Computing - Lecturenotes SoSe2016}
\author{Julius Hülsmann}
	\pagestyle{fancy}
\begin{document}
  \maketitle
  \tableofcontents
  \newpage

\section{Repetition}
\subsection{Containerization}
encapsulate application together with its operating environment
\subsection{Software-defined Networking}
Encapsulation: gives access to configuration of "ntwork-layer" without being confronted with the hardware. 


Separation into {\bf Control Plane} and {\bf Data Plane}
\subsubsection{Control Plane}
Make decision on data's destination\\
The control plane functions include the system configuration, management, and exchange of routing table information
\subsubsection{Data Plane}
Also known as Forwarding Plane\\
Forwards traffic to the next hop along
\subsubsection{Example from sdntutorials.com}
The protocol or application itself doesn't really determine whether the traffic is control, management, or data plane, but more importantly how the router processes it. Consider a 3 router topology with routers R1, R2 and R3. 
Lets say a Telnet session is established from R1 to R3. On both of these routers the packets need to be handled by the control/management plane. However from R2's perspective this is just data plane traffic that is transiting between its links.


\subsection{Continuous Integration}
Improvement, Automatization of the Deployment organization

\subsubsection{Deployment Pipeline}
Menge von Validierungen, die eine Software auf ihrem Weg zur Veröffentlichung bestehen muss
\subsection{Continuous Delivery}
Improvement of the Deployment organization\\
Wichtiger Begriff: Deployment pipeline\\
Laut Wikipedia \glqq Weiterentwicklung\grqq von Continuous Integration, indem extends CI by making sure the software checked in on the mainline is always in a state that can be deployed to users and makes the actual deployment process very rapid.
\newpage
\section{Dev-Ops}
\subsection{Definition and Goals}
\begin{itemize}
	\item Set of practices
	\item [Aim:] 
	\item Reduce Time between committing and normal Use of the software
	\item high Quality
\end{itemize}
Practices
\begin{itemize}
	\item Iterative $\Rightarrow$ ok 4 changing requirements, fast delivery and high-quality
	\item small teams, improve collabortation dev- \& ops, reduce communication 
	\item frequent commits, automatic deployment \& testing, early problem detection
\end{itemize}
%Knight-Capital
Influences
\begin{itemize}
	\item the way systems are built	
	\item (incl.) the Organization of Teams
	\item the structure of the built system
\end{itemize}
\grafik{../zusa2/DevOps/01.png}{}

\newpage
\section{Microservices}
\subsection{Definition and Motivation}
Way of developping: 
\begin{itemize}
	
\item architectural style\\
\item single application = Set of small services (own processes, programming language, databases \item technologies); communication often via EXTERNALIZABLE APIS (often lightweight http)  \\
\item independently deployable, automatic \\
\item are minimum of centralized management \\
\item – no direct linking, no direct reads of another team’s data store,
– no shared-memory model, no back-doors whatsoever.
– The only communication allowed is via service interface calls over the network.
\end{itemize}
Frage: Wie weit verschachtelt (depth) sind die Microservices?

Unterschied zu Monoliths (all functionallity in one application, scaled by replication)
Microservice: (Scales by distributing services to replicas; Replication as needed)


\subsection{Characteristics of Microservices}
\begin{enumerate}
	\item Componentization via Services\\
	Component = independant, upgradable\\
	\qquad Explicit dependencies
\\$\left . \right .$\qquad No specific assumptions on execution environment
\\$\left . \right .$\qquad Different possible realizations (programming lang.., code structure)
\\$\left . \right .$\qquad Composable from other software components\\
	Libraries = Components linked into program; in-memory-function calls\\
	Service = out of process, Communicate f.i. via http
		\item Organized around Business Capabilities \\
		Wenn das Team nach Gruppen sortiert ist Auswirkungen auf Program. Besser ggf mix
		\item Products not Projects\\
		you build, you run it vs. nur build
	\item Smart end points and dumb pipes\\
	as decoupled and as cohesive as possible, filters in Unix sense
	\item Decentralized Governance\\
	Centralized $\Rightarrow$ standartize technology used. Not good
	\item Decentralized Data Management (= multiple DB with different content)
	\item Infrastructure Automation\\(= Automated, continuous deployment and testing processes)
	\item Design for failure\\
	design to tolerate failures of other services (could fail due to unavailability of the supplier)\\
	constantly reflect on how service failures affect the user experience
	\item Evolutionary Design\\
	granular release planning.
Monolithic: build and deployment of the entire application. \\
microservices, only need to redeploy the service(s) you modified
\end{enumerate}


\newpage
\section{Continuous Engineering}




\grafik{../zusa2/DevOps/02.png}{}
\paragraph{Probleme}
Deployment: Es kann sich ändern, wie neu deployter Microservice andere - benutzt und benutzt wird.
Any deployment team has to be able to deploy at any time
it takes time to replace one VM
Service to clients must be maintained

\subsection{Strategies}
\subsubsection{All-or-nothing (alle Testenden Involved)}
\begin{itemize}
	\item Blue-Green or Big Flip Deployment
	leave N VMs with version A as they are, 
	allocate and provision N VMs with version B, switch to version B; once stable and release VMs with version A.
	\item Rolling upgrade
	()allocate a new VM with version B, release one VM with version A. Repeat N times)
\end{itemize}
\begin{table}[h]
	\centering
	\label{my-label}
	\begin{tabular}{|l|l|}
		\hline
		Blue-Green                           & Rolling Upgrade      \\ \hline
		Only one version available at a time & 2 versions available \\ \hline
		2N VMs                               & N+1 VMs              \\ \hline
		Rollback easy                        &                      \\ \hline
	\end{tabular}
\end{table}

\subsubsection{Partial strategies - kleine Gruppe enduser}
\begin{itemize}
	\item Canary testing
	few servers of a new version, 
	running in production in order to perform live testing in the real environment
	Testers. 
	organization-based (dog-fooding) / geographically based / Rand
	Rout the specified requests to the testing image (e.g. Load balancer), observe
	\item A/B Testing
	Both systems in parallel, tested by someone wo wants new features
	\item – Chaos / LATENCY Monkey: randomly kills VMs in production
\end{itemize}
\newpage
\section{Doofe Case studies}
Run virtual plans for each branch – avoids plan drift\\
– Branch awareness supported by Bamboo

\subsection{1}
\subsection{2}
\end{document}