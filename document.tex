%\documentclass[a4paper,12pt]{article}%report
 
 \documentclass[a4paper,12pt]{article}%report
 % das hier ist eine font
 %\usepackage{newcent}
 
\usepackage{amsthm}
\usepackage[ngerman]{babel}     
\usepackage[utf8]{inputenc}      
\usepackage {amsmath}
\usepackage{bbm}
\usepackage{graphicx}
\usepackage{color}
\usepackage{hyperref}
\usepackage{tabularx}
\usepackage{booktabs}% http://ctan.org/pkg/booktabs
\newcommand{\tabitem}{~~\llap{\textbullet}~~}
\usepackage[arrow, matrix, curve]{xy}

\usepackage{multirow}
 \usepackage[table,xcdraw]{xcolor}
\usepackage{textpos}

%\newenvironment{ex}{begdef}{enddef}
% Theorem und Satz           
\newtheorem{thm}{Theorem}[section]
\newtheorem{myDef}[thm]{Definition}
% \newtheorem{thm}{Satz}[section]
\newtheorem{ex}{Aufgabe}[section]
% \newtheorem{myDef}[thm]{Definition}
\newtheorem{myExmpl}[thm]{Beispiel}
\newtheorem{myCmmt}[thm]{Bemerkung}
\newtheorem{lem}[thm]{Lemma}
\newtheorem{prop}[thm]{Proposition}
\newtheorem{myKor}[thm]{Korollar}
\newtheorem{myV}{Veranschaulichung}
\newtheorem{myLemma}[thm]{Lemma}

\usepackage[table,xcdraw]{xcolor}

\newcommand{\beg}[1]{\textsc{#1}}
\newcommand{\openEx}[1]{ {\bf Aufgabe #1:}}
\newcommand{\openAns}{ {\bf Antwort:~}}

% neue eigene shortcuts
\newcommand{\N}{\mathbbm N}
\newcommand{\K}{\mathbbm K}
\newcommand{\R}{\mathbbm R}
\newcommand{\A}{\mathcal{A}}
\newcommand{\B}{\mathcal{B}}
\newcommand{\F}{\mathcal{F}}
\newcommand{\M}{\mathcal{M}}
\newcommand{\HH}{\mathcal{H}}
\newcommand{\U}{\mathcal{U}}
\newcommand{\PP}{\mathbbm P}
%\newcommand{\r}{\mathbbm R}
\newcommand{\C}{\mathbbm C}
\newcommand{\Z}{\mathbbm Z}
\newcommand{\zuZeigen}{ \mathrm{Z\kern-.5em\raise-0.4ex\hbox{Z}}}
\newcommand{\hnot}[1]{{\it #1}}
\newcommand{\Hnot}[1]{{\bf #1}}

\newcommand{\refFootnote}[1]{\footnote{#1}}
\definecolor{rahmen}{rgb}{.7,0.9,0.8}    
\definecolor{grund}{gray}{.99}           
\definecolor{schrift_box}{cmyk}{0.9,0.7,0.8,0}   
\definecolor{whites}{cmyk}{0, 0, 0,0}   
\definecolor{schrift_normal}{gray}{0}   
\definecolor{ekelig}{cmyk}{0.1,0.9,0.9,0}   
\definecolor{ekeligHintergrund}{cmyk}{0.1,0.1,0.9,0}   
  
  \newcommand{\defWord}[1]{{\rm #1}}
\newcommand{\openBox}[1]{
   \color{grund}
  
  \fcolorbox{rahmen}{grund}{\fbox{ \color{schrift_box}\parbox{\dimexpr \linewidth - 2\fboxrule - 2\fboxsep}{#1}}}
  \color{schrift_normal}}
  
\newcommand{\marker}[1]{
   \color{ekelig}
\fcolorbox{whites}{ekeligHintergrund}{
    TODO: #1
}
    \color{schrift_normal}}
  
\newcommand{\openBoxx}[2]{
   \color{grund}
  
  \fcolorbox{rahmen}{grund}{\fbox{ \color{schrift_box}\parbox{\dimexpr \linewidth - 2\fboxrule - 2\fboxsep}{#1}}{#2}}
  \color{schrift_normal}}
  
% Das hier ist nur so einnkommentiert, da ich auf OS X nicht herausgefunden 
% habe, wie man mathbbm zum Laufen kriegt.
%\newcommand{\N}{N}
%\newcommand{\K}{K}
%\newcommand{\R}{R}%
%\newcommand{\C}{C}
%\newcommand{\Z}{Z}
%\newcommand{\zuZeigen}{Z}

\newcommand*{\nametag}[1]{%
  \stepcounter{equation}%
  \tag{\tagtext{#1}\tagcomma\tagnumber{\theequation}}%
}
\makeatletter% <-- anklicken liefert Erklärung
\newcommand*{\eqnumref}[1]{%
  \begingroup
    \let\tagtext\@gobble
    \let\tagcomma\relax
    \eqref{#1}%
  \endgroup
}
\newcommand*{\eqtextref}[1]{%
  \begingroup
    \let\tagcomma\relax
    \let\tagnumber\@gobble
    \ref{#1}%
  \endgroup
} 


\newcommand{\grafik}[2]{\begin{figure}[!htb]
		\noindent\includegraphics[width=\linewidth,height=\textheight,
		keepaspectratio]{#1}
		\caption{\textrm{#2}}%
	\end{figure}}
\usepackage{cite}


\newcommand{\grafikM}[2]{\begin{figure}[!htb]
		\noindent\includegraphics[width=12.5cm,height=6.5cm]{#1}
		\caption{\textrm{#2}}%
	\end{figure}}
	\usepackage{cite}
	

\newcommand{\grafikMNon}[2]{\begin{figure}[!htb]
		\noindent\includegraphics[width=\linewidth]{#1}
		\caption{\textrm{#2}}%
	\end{figure}}
	\usepackage{cite}

\newcommand{\grafikMS}[2]{\begin{figure}[!htb]
		\noindent\includegraphics[height=4cm]{#1}
		\caption{\textrm{#2}}%
	\end{figure}}
	\usepackage{cite}
	

\newcommand{\grafikMHalf}[2]{\begin{figure}[!htb]
		\noindent\includegraphics[width=12.5cm,height=2.5cm]{#1}
		\caption{\textrm{#2}}%
	\end{figure}}

\usepackage{cite}




\newcommand{\newDef}[1]{ {\bf #1} }

\newcommand{\oldDef}[1]{ {\it #1} }




\usepackage{blindtext}

\usepackage{fancyhdr}%Kopf- und Fußzeile
\renewcommand{\headrulewidth}{1pt} %Linie oben
\fancyhf{}
\fancyhead[L]{\leftmark} %Kopfzeile links bzw. innen
\fancyhead[R]{\thepage} %Kopfzeile rechts bzw. außen


%Das hier ist neu dazugekommen für automatische Wörtertrennung.
\usepackage{xspace}



\setlength\parindent{0pt}

\newif\ifcomment



% zum abaendern wenn die Erklaerungen ausgeblendet werden sollen
\newcommand{\invisible}[1]{#1}


\newcommand{\lecture}[2]{\color{black} 
	
	\begin{textblock*}{3cm}(-2.4cm,0.125cm)
		{\bf Lecture {#1}}
	\end{textblock*}
	
	\begin{textblock*}{3cm}(-2.4cm,0.625cm)
		#2
	\end{textblock*}
}
\newcommand{\tutorial}[2]{\color{black} 
	
	\begin{textblock*}{3cm}(-2.4cm,0.125cm)
		{\bf Tutorial {#1}}
	\end{textblock*}
	
	\begin{textblock*}{3cm}(-2.4cm,0.625cm)
		#2
	\end{textblock*}
}



%cpp stuff

%\setmonofont{Consolas} %to be used with XeLaTeX or LuaLaTeX
\definecolor{bluekeywords}{rgb}{0,0,1}
\definecolor{greencomments}{rgb}{0,0.5,0}
\definecolor{redstrings}{rgb}{0.64,0.08,0.08}
\definecolor{xmlcomments}{rgb}{0.5,0.5,0.5}
\definecolor{types}{rgb}{0.17,0.57,0.68}

\usepackage{listings}
\lstset{language=C++,
	captionpos=b,
	%numbers=left, %Nummerierung
	%numberstyle=\tiny, % kleine Zeilennummern
	frame=lines, % Oberhalb und unterhalb des Listings ist eine Linie
	showspaces=false,
	showtabs=false,
	breaklines=true,
	showstringspaces=false,
	breakatwhitespace=true,
	escapeinside={(*@}{@*)},
	commentstyle=\color{greencomments},
	morekeywords={partial, var, value, get, set},
	keywordstyle=\color{bluekeywords},
	stringstyle=\color{redstrings},
	basicstyle=\ttfamily\small,
}


% \newcommand{\erklaerung}[1]{}
%opening
\title{Advanced Enterprise Computing - Lecturenotes SoSe2016}
\author{Julius Hülsmann}
	\pagestyle{fancy}
\begin{document}
  \maketitle
  \tableofcontents
  \newpage

\section{Repetition}
\subsection{ACID}

{\bf Atomiticity}: Either  entire Transaction  executed or aborted. Update all Replicas or none\\
{\bf Consistency}: does not mean Data-Consistency but that the transaction produces consistent changes, client-centric / Data-centric, read-my-writes etc..\\
{\bf Isolation}: Transactions are isolated from one another\\
{\bf Durability}: Once the transaction is ready (commits) it remains.

~\\
Both the Atomitcity and the Isolation are managed by the {\bf Transaction Manager}
\begin{itemize}
 \item Aquires locks on behalf of the transactions
 \item guarantees serializable execution (= strongest form of isolation; outcome = outcome in case sequentially executed). \\
 guaranteed by use of 2PL
\end{itemize}

\begin{table}[h]
 \centering
% \caption{Protpocols}
% \label{my-label}
\begin{tabular}{llll}
\rowcolor[HTML]{ECF4FF} 
\textbf{name} & \textbf{protocol's name}     & \textbf{bsp implementations}  & \textbf{distrib. syst} \\
\rowcolor[HTML]{F7F7F7} 
Atomacity     & atomic committement prot. & 2PC        & easy, expensive                          \\
\rowcolor[HTML]{EFEFEF} 
Isolation     & concurrency control protocol & 2PL, Snapshot isolation   & problematic          
\end{tabular}
\end{table}


\subsection{CAP}
CAP = Consistency vs. Avaiability in case of Partitioning (Replication) Either one has to choose consistency or availability.

Gives information on the system's behavior in case of a system error.

\subsection{PACELC}
PACELC: partitioning: Avaibaility/Consistency else Latency/Consistency

In case an update request arrives and the data is replicated and the system is working properly,
there is a time difference between the moment the first replica receives the update and the other replicas are informed.
Now it's the question whether to 
\begin{enumerate}
 \item immeadiately commit the new data {\it (chose the least latency)} or 
 \item whether not to respond until the 2nd replica respons {\it (chose consistency)} 
\end{enumerate}
\grafik{zusa2/00.png}{Cap and PACELC}

\subsection{Consistency}
2 Dimensions:
\begin{itemize}
	\item {\bf Staleness} (how much is a given replica lagging behind) vs. {\bf Ordering} (how much does the operation serializable order deviate among replicas)
	\item {\bf Data-centric} and {\bf Client-centric} consistency
\end{itemize}

\subsubsection{Ordering and Data-centric consistency}
\begin{itemize}
\item Sequential Consistency 
\item Causal consistency
\item Eventual Consistency
\end{itemize}
\subsubsection{Ordering and Client-centric consistency}
\begin{itemize}
	\item Monotonic Reads (never receive older values than previously read)
	\item Read-my-writes (not older than previously written)
	\item Write follows reads (only update Replicas that have at least got read entity)
	\item Monotonic Writes (update of one client: always executed sorted by time)
\end{itemize}
\newpage
 
 \section{Replication and State Management\\ (25.04. - 09.05.)}


\subsection{Motivation and Background}

\subsubsection{Replication}
\paragraph{Definition - Replication}
Process of maintaining multiple Copies of an Entity (Data / Process / File ...)

\paragraph{Advantages of Replication in General}
\begin{itemize}
	\item {\it System Availability / Fault tolerance / Security} in case \\
	A \quad Server fails \\
	B \quad Data is corrputed (vote against data, Byzantine)
	\item {\it Performance / Scalability } \\
	A \quad Workloads are spread across distributed Replicas \\
	B \quad Geodistribution for processing demands in client's proximity
\end{itemize} 

\paragraph{Disadvantages of Replication in General}
\begin{itemize}
	\item Consistency vs. Performance
\end{itemize}
\paragraph{Kinds of Replication}
There are the following three different decisions one needs to take for developing a suitable Replication design:
\begin{enumerate}
 \item \glqq Phyiscal Replication\grqq (A), (B) or (C)
 \item Defined access and update Mechanisms(Synchronous/Asynchronous PrimaryCopy/Update Everywhere)
 \item Where to put the Replicas (Geo-distributed?)
\end{enumerate}


\paragraph{decision 1) Replication}
In general there are the following kinds of \glqq physical\grqq ~Replication. We do only consider (B).
\grafik{src/Grafik01}{(A) Improves the availability only in case the Server Replicas use a replica cache coherence mechanism
	(B) Improves the availability. Usually ment by the term \glqq Replication\grqq.}
\paragraph{decision 2) Replication Strategies}~\\
When\qquad Sync./eager vs. Async./lazy\qquad \qquad\qquad\qquad update propagation

Where\qquad Primary Copy/master vs. upd. Everywhere/group\qquad - location

\paragraph{decision 3) Replicas' location}
There are permanent Replicas, they can be geo-distributed or within the LAN

Server-initiated (e.g. Push-Cache) or Client-initiated (e.g. Cache) temporary Replicas exist, too.

Push-Cache: Server record request-history

$\Rightarrow$ Tradeoff between consistency and latency is accentuated

\paragraph{What happens to ACID in case of Replication?}
Atomicity can be guaranteed using 2PC (but expensive)
Problem: Serialization order must be the same at all replicas.



\ifcomment
{\bf Synchronous} 


\begin{enumerate}
 \item 1 propagate Data  to everybody
 \item Wait until everybody responded 
 \item commit        
\end{enumerate}

 {\color[HTML]{9AB299} 
- ACID properties apply to all copy updates (no Inconsistencies)}    
{\color[HTML]{A38C8C} -  High response time (high execution time, response time)\\
- Availability (in case one Copy fails)}


{\bf Asynchronous}
\begin{enumerate}
\item Update local copy
 \item Commit
 \item propagate Data  to everybody
\end{enumerate}
 {\color[HTML]{9AB299} - Response Time\\ - Availability}    
 
{\color[HTML]{A38C8C} 
 
- Data inconsistency
(local read does not always
1 Update local copy
return the latest value)
- No guarantee that the changes
arrive at each copy
- Replication is not guaranteed (it is possible that the changes do not reach the replicas at all)
}



{\bf Primary Copy}
  Each update is initiated by the Promary Copy


 {\color[HTML]{9AB299} 
 \begin{itemize}
  \item (OWN) The order of update can be controled by the primary copy
  \item No inter-site connection is necessary. All information cross the primary copy
  \item There's always one page which has got the latest version of the data (primary copy)
 \end{itemize}

 }   
{\color[HTML]{A38C8C}
 \begin{itemize}
  \item High latency due to high load at Primary Copy
  \item Single Point of Failure
  \item Reading local copy does not always yield the latest information
 \end{itemize}
 }

{\bf Update everywhere}

 {\color[HTML]{9AB299} 
 \begin{itemize}
  \item There is not a single point of failure
  \item load is evenly distributed
  \item Any site can run a transaction 
 \end{itemize}

 }   
{\color[HTML]{A38C8C}
 \begin{itemize}
  \item Concurrent updates may cause conflicts; No update order can be guaranteed
  \item Copies must be synchronized. (Workload)
 \end{itemize}
 }
 
 
 Synchronous Primary Copy
 
 {\color[HTML]{9AB299} 
 \begin{itemize}
  \item {\bf Consistency is} guaranteed.
  \item Updates {\bf don't need to be coordinated}; The transaction order can be guaranteed
  \item {\bf no deadlocks}
 \end{itemize}

 }   
{\color[HTML]{A38C8C}
 \begin{itemize}
  \item Longest response time; only useful in case of few updates (primary Copy = Bottleneck)
  \item Low availability: Single Point of Failure; in case a replica fails, the entire system may be blocked
  \item Read-only Replicas -- nearly useless
  \item Scalability Problems duee to primary copy Bottleneck
  \item[$\Rightarrow$] Not used in practice, therefore no Protocol attatched
 \end{itemize}
 }
 
 
 Synchronous Update Everywhere
 
 {\color[HTML]{9AB299} 
 \begin{itemize}
  \item Elegant, Symmetric solution
  \item {\bf Consistency} is guaranteed
  \item {\bf High fault tolerance}
 \end{itemize}

 }   
{\color[HTML]{A38C8C}
 \begin{itemize}
  \item {\bf Performance} Very high numbers of messages involved -- $2N$ Messages + time for 2PC
  \item Transaction response time is very long
  \item {\bf The system will not scale} (get a lot of nodes) because with an increase in the number of nodes comes a higher probability of getting into a deadlock.
  \item Deadlocks may occur. (the rate depends on the amount of transactions per second, the size of the database and the number of nodes)
  
  \item Updates need to be coordinated (update order, parallel updates)
  \item Low availability
 \end{itemize}
 }
 

 Asynchronous Primary Copy
 
 {\color[HTML]{9AB299} 
 \begin{itemize}
  \item Updates don't need to be coordinated
  \item {\bf Short Response time}
 \end{itemize}

 }   
{\color[HTML]{A38C8C}
 \begin{itemize}
 \item {\bf Fault tolerance is limited}, temporary data {\bf inconsistenceis} arise. Inconsistencies 
 \item Local copies may not be up-to-date (due to delay)
 \item Single Point of Failure,
 \item Low write availability
 \end{itemize}
 }

 
 
 Asynchronous update Everywhere
 
 {\color[HTML]{9AB299} 
 \begin{itemize}
  \item {\bf Shortest Response time}
  \item {\bf heigh fault tolerance} Not a single point of failure, no centrailized coordination
  \item High availability
 \end{itemize}

 }   
{\color[HTML]{A38C8C}
 \begin{itemize}
 \item Updates need to be coordinated
 \item {\bf Reconciliation} and inconsistenceis, updatese can be lost 
 \end{itemize}
 }
 
 
 
 
 \fi
 
 \newpage
 \subsection{managing Replication}
 
 
 
 
 \grafik{zusa2/01.png}{Different strategies: Advantages, Disadvantages, Properties}
 \grafik{zusa2/02.png}{Different strategies: Algorithm in detail}
 
 \paragraph{Synchronous Update Everywhere}
 
 {\bf Protocol}
 Assumption: All sites (Replicas) Contain the same data. Behavior if Transaction is to be executed
 \begin{itemize}
  \item Local use of 2PL for the following steps:
  \item READ only one site, in case the reading fails (timeout), read another copy
  \item WRITE at all sites (distributed locking protocol). This means that all copies of the data item need to lock the item (REQUEST, OBTAIN LOCK, ACK)\\~. \\
  IF one site rejects, ABORT. \\
  ADD All site not responding to a list of {\it missing writes}.
  \item VALIDATE (=commit) the transaction at the end, this means\\~ \\ 
  IF NOT ALL the servers {\it missing writes} are down: ABORT \\
  IF NOT ALL the servers that {\it accepted} are still available: ABORT \\
  OTHERWISE Commits
 \end{itemize}

 
 
 \begin{itemize}
  \item[$\Rightarrow$] Guarantees behaviour like if the sites were not replicated.
  \item[$\Rightarrow$]  Execution is serializable
  \item[$\Rightarrow$]  all Reads access the latest version
 \end{itemize}

 {\bf Extensions for coping with failures} \\
 1. Site failure (reduces the avalability) \\
 2. Behavior after the Site that failed is online again (outdated data available)
 
 
 
 Optimization: Most ideas based on Quorums
 
 \ifcomment
 \paragraph{Asynchronous Primary Copy}
 \begin{itemize}
  \item BOT: Primary Copy receives transaction
  \item EOT: Primary Copy applies transaction (2PL locally) and COMMITS
  \item The changes are propagated to all Replicas
 \end{itemize}

 
 
 \paragraph{Asynchronous Update Everywhere}
 \begin{itemize}
  \item BOT, W, R, ..., EOT executed locally (2PL locally) (locally committed) 
  \item Changes are propagated to all other replicas
  \item Reconciliation (Last update, site priority, value priority, try to merge, rm not important one, priority shemes
 \end{itemize}

 
 
 
 
 
 
 
 
 
 
 
 
 \fi
 
 
 
 
 \paragraph{Quorums}
 kind of a middleground between syncrhonous and Asynchronous updates.
 
 Reads contact more than one Replica
 
 Write contacts a quorum of Replicas.
 
 Rules:
 
 Read at least $1$ Replica that has received the latest update: $R+W > N$,\\
 The minimum amount of writes must be greater than half of the amount of replicas $\frac{W}{2} > N$
 
 Quorums that don't follow these rules are called {\bf sloppy Quroums}. (Dynamo + Cassandra)
 
 Used to trade off read and write latency
 
 
 
 
 \paragraph{Different views on the subject}
 The solution to Replication strategies in {\it Database-POV} and {\it Distributed Systems-POV} differ, but have converged.
 
 
 Distributed System
 \begin{itemize}
  \item Set of Services implemented by Server Processes, invoked by client processes
  \item Each server has got a local state
  \item Group of servers / group communication -- helps to reduce complexity
  \item Group communciation primitives: provide 1toMany communication. Example:  Atomic Broadcast (ACAST), View-Syncrhonous Broadcast (VSCAST) 
 \end{itemize}


\grafik{zusa2/03.png}{Distributed system perspective}
\ifcomment
 \paragraph{Active Replication}
 ABCAST guarantees atomacity, sent to everybody 
 All Servers execute the transaction in the order they arrive
 Everybody informs the client (which typically takes the first answer).
 
 + Simple
 + failure-transparency; Failures completely hidden from client
 - determinism constraint (no multicore servers) c
 - complexity shifted to middleware layer (communication)
 
 
%  Asynchronous Primary Copy
 \paragraph{Passive Replication}
 Client -> Primary which executes the Transaction and sents the result to the backups
 
  Connunication has to guarantee right order (VSCAST)
 
 By
doing so, no determinism constraint is necessary on the execution of
invocations.

1 Client sents request
2 Primary applies directly
3 primary informs other servers (VSCAST)
4 Primary sents answer to the client

+ tolerates non-deterministic servers (Threaded)
+ Little processing power used
- High reconfiguration cost if primary fails

%  Syncrhonous Update Everywhere
 \paragraph{Semi-Active Replication}
Combination of both

Each time Replicas have to do a non-deterministic decision, one replica (called Leader) broadcasts the result to the other replicas(followers).
 
 
 
 \fi
 
 
 
 
 
 \subsubsection{Reconciliation}
 Who is responsible; When?\\
 Alternative: Multiversion systems like DynamoDb
 
 pre-arranged patterns:
 \begin{itemize}
 	\item Last update wins (newer updates preferred over old ones)
 	\item  Site priority (preference to updates from headquarters)
 	\item  Largest value (the larger transaction is preferred
 \end{itemize}
 ad-hoc decision making procedures
 \begin{itemize}
 	\item Combination
 	\item Analyze and remove not important ones
 	\item ...
 \end{itemize}
 
 
 
 
 
 
 
 
 
 
 \subsubsection{Conclusion}
 if scalable $\Rightarrow$ Conflicts have to be resolved / appear
 
 \newpage
 \subsection{Paxos}
 case: Conflicts cant be tolerated, using a scalable system. Not applicable for Byzantine failure \\
 Rules:
 \begin{enumerate}
\item exactly one value is chosen
\item non-triviality (chose one proposed val)
\item liveness (failure tolerant in case less than half fail)
 \end{enumerate}
 Not nPC  because may violate (Rules 1 and 3)
 
 Client, Proposer, Acceptor, Learner
 
\subsubsection{extensions}
\paragraph{Multi-Paxos}

\paragraph{Fast Paxos}
\paragraph{Generalized Paxos}
\subsubsection{Summary Paxos}
\begin{itemize}
	\item[$\Rightarrow$] guarantees agreement in the presence of failures, Safety is always preserved
	\item[$\Rightarrow$]  Conditions to affect liveness are hard to provoke
	\item[$\Rightarrow$]  Uses same number of message rounds as 2PC
\end{itemize}

\newpage
\section{Lecturenotes}
\lecture{05?}{2016-05-09} start @ 81
Für Donnerstag paper mitbringen und Paxos anschauen.

\paragraph{Paxos} (Represent as State-machine) - P. 77


Proposer


Phase 1
- Proposer choses Number largr than any value chosen before by Propposer. 
- Broadcast the integer {\it prepare(n)}, e.g. prepare(50)

Acceptors
a) Not respond at all
b) {\it recject} Reject, in case a higher value has been accepted. 50 < something
b) {\it prommise(n)} in case 50 > everything. Also Send everything that has already been accepted.

If prposer receives majority of prommise respons, -> proceed to Phase 2 ELSE -> Phase 1

Phase 2
- Check whether any <n, value> have been returned. 
- YES: take max n's value
- accept (n, value)

\paragraph{Xtensions Paxos}
	
	{\bf Multi-paxos}
	Determine Leader once
	Stay in phase 2, attatch the leader identifier
	Leader is the one to accept values
	
	{\it Purpose:  Optimize Speed} (get rid of the first phase, Master-Slave setup)
	
	
	{\bf Fast Paxos}
	
	{\bf Generalized Paxos}
	- Assumption: The execution order does not matter.

\paragraph{CRDT}
Conflict free / Communitive replicated Datatypes

Some operations are commutative, others not.

State- Based vs. Operation based.

theoretically it is possible to converge them but ... practice




IDEA
INTEGER  - example: 
e.g. not store int values but operations (increment / decrement))

SET - example

State - based Set


\section{Begriffe und Abkürzungen}

\begin{description}
	\item[Replication] Strategy to maintain mutiple copies of an entity on multiple Servers.
	\item[Replica] 
	\item[CRDT] {\it conflict-free replicated data}
	\item[Paxos] 
	\item[Commit] In case a Transaction commits, it is ready.
	\item[Concurrency control protocol] guarantees isolation of Transactions
	\item[2PL] Two phase locking (one concurrency control protocol)
	\item[Snapshot Isolation] other cuncurrency control protocol implementation
	\item[atomic commitement protocol] guarantees atomaticity
	\item[2PC] Two phase Commit
	\item[Transaction Manager] Middleware Component; Manages Atomacity and Isolation of Transactions
	\item[ACID]  Atomacity + Consistency + Isolation + Durability
	\item[serializability] a plan of executing multiple transactions in pseudo- parallel is called serializable in case the parallel execution comes to the same result as executing the transactions one after the other
	\item[distributed locking protocol] 2PL z.b.? ??? Paxos??ß
	
\end{description}
\end{document}