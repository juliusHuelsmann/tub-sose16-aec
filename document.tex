%\documentclass[a4paper,12pt]{article}%report
 
 \documentclass[a4paper,12pt]{article}%report
 % das hier ist eine font
 %\usepackage{newcent}
 
\usepackage{amsthm}
\usepackage[ngerman]{babel}     
\usepackage[utf8]{inputenc}      
\usepackage {amsmath}
\usepackage{bbm}
\usepackage{graphicx}
\usepackage{color}
\usepackage{hyperref}
\usepackage{tabularx}
\usepackage{booktabs}% http://ctan.org/pkg/booktabs
\newcommand{\tabitem}{~~\llap{\textbullet}~~}
\usepackage[arrow, matrix, curve]{xy}

\usepackage{multirow}
 \usepackage[table,xcdraw]{xcolor}
\usepackage{textpos}

%\newenvironment{ex}{begdef}{enddef}
% Theorem und Satz           
\newtheorem{thm}{Theorem}[section]
\newtheorem{myDef}[thm]{Definition}
% \newtheorem{thm}{Satz}[section]
\newtheorem{ex}{Aufgabe}[section]
% \newtheorem{myDef}[thm]{Definition}
\newtheorem{myExmpl}[thm]{Beispiel}
\newtheorem{myCmmt}[thm]{Bemerkung}
\newtheorem{lem}[thm]{Lemma}
\newtheorem{prop}[thm]{Proposition}
\newtheorem{myKor}[thm]{Korollar}
\newtheorem{myV}{Veranschaulichung}
\newtheorem{myLemma}[thm]{Lemma}

\usepackage[table,xcdraw]{xcolor}

\newcommand{\beg}[1]{\textsc{#1}}
\newcommand{\openEx}[1]{ {\bf Aufgabe #1:}}
\newcommand{\openAns}{ {\bf Antwort:~}}

% neue eigene shortcuts
\newcommand{\N}{\mathbbm N}
\newcommand{\K}{\mathbbm K}
\newcommand{\R}{\mathbbm R}
\newcommand{\A}{\mathcal{A}}
\newcommand{\B}{\mathcal{B}}
\newcommand{\F}{\mathcal{F}}
\newcommand{\M}{\mathcal{M}}
\newcommand{\HH}{\mathcal{H}}
\newcommand{\U}{\mathcal{U}}
\newcommand{\PP}{\mathbbm P}
%\newcommand{\r}{\mathbbm R}
\newcommand{\C}{\mathbbm C}
\newcommand{\Z}{\mathbbm Z}
\newcommand{\zuZeigen}{ \mathrm{Z\kern-.5em\raise-0.4ex\hbox{Z}}}
\newcommand{\hnot}[1]{{\it #1}}
\newcommand{\Hnot}[1]{{\bf #1}}

\newcommand{\refFootnote}[1]{\footnote{#1}}
\definecolor{rahmen}{rgb}{.7,0.9,0.8}    
\definecolor{grund}{gray}{.99}           
\definecolor{schrift_box}{cmyk}{0.9,0.7,0.8,0}   
\definecolor{whites}{cmyk}{0, 0, 0,0}   
\definecolor{schrift_normal}{gray}{0}   
\definecolor{ekelig}{cmyk}{0.1,0.9,0.9,0}   
\definecolor{ekeligHintergrund}{cmyk}{0.1,0.1,0.9,0}   
  
  \newcommand{\defWord}[1]{{\rm #1}}
\newcommand{\openBox}[1]{
   \color{grund}
  
  \fcolorbox{rahmen}{grund}{\fbox{ \color{schrift_box}\parbox{\dimexpr \linewidth - 2\fboxrule - 2\fboxsep}{#1}}}
  \color{schrift_normal}}
  
\newcommand{\marker}[1]{
   \color{ekelig}
\fcolorbox{whites}{ekeligHintergrund}{
    TODO: #1
}
    \color{schrift_normal}}
  
\newcommand{\openBoxx}[2]{
   \color{grund}
  
  \fcolorbox{rahmen}{grund}{\fbox{ \color{schrift_box}\parbox{\dimexpr \linewidth - 2\fboxrule - 2\fboxsep}{#1}}{#2}}
  \color{schrift_normal}}
  
% Das hier ist nur so einnkommentiert, da ich auf OS X nicht herausgefunden 
% habe, wie man mathbbm zum Laufen kriegt.
%\newcommand{\N}{N}
%\newcommand{\K}{K}
%\newcommand{\R}{R}%
%\newcommand{\C}{C}
%\newcommand{\Z}{Z}
%\newcommand{\zuZeigen}{Z}

\newcommand*{\nametag}[1]{%
  \stepcounter{equation}%
  \tag{\tagtext{#1}\tagcomma\tagnumber{\theequation}}%
}
\makeatletter% <-- anklicken liefert Erklärung
\newcommand*{\eqnumref}[1]{%
  \begingroup
    \let\tagtext\@gobble
    \let\tagcomma\relax
    \eqref{#1}%
  \endgroup
}
\newcommand*{\eqtextref}[1]{%
  \begingroup
    \let\tagcomma\relax
    \let\tagnumber\@gobble
    \ref{#1}%
  \endgroup
} 


\newcommand{\grafik}[2]{\begin{figure}[!htb]
		\noindent\includegraphics[width=\linewidth,height=\textheight,
		keepaspectratio]{#1}
		\caption{\textrm{#2}}%
	\end{figure}}
\usepackage{cite}


\newcommand{\grafikM}[2]{\begin{figure}[!htb]
		\noindent\includegraphics[width=12.5cm,height=6.5cm]{#1}
		\caption{\textrm{#2}}%
	\end{figure}}
	\usepackage{cite}
	

\newcommand{\grafikMNon}[2]{\begin{figure}[!htb]
		\noindent\includegraphics[width=\linewidth]{#1}
		\caption{\textrm{#2}}%
	\end{figure}}
	\usepackage{cite}

\newcommand{\grafikMS}[2]{\begin{figure}[!htb]
		\noindent\includegraphics[height=4cm]{#1}
		\caption{\textrm{#2}}%
	\end{figure}}
	\usepackage{cite}
	

\newcommand{\grafikMHalf}[2]{\begin{figure}[!htb]
		\noindent\includegraphics[width=12.5cm,height=2.5cm]{#1}
		\caption{\textrm{#2}}%
	\end{figure}}

\usepackage{cite}




\newcommand{\newDef}[1]{ {\bf #1} }

\newcommand{\oldDef}[1]{ {\it #1} }




\usepackage{blindtext}

\usepackage{fancyhdr}%Kopf- und Fußzeile
\renewcommand{\headrulewidth}{1pt} %Linie oben
\fancyhf{}
\fancyhead[L]{\leftmark} %Kopfzeile links bzw. innen
\fancyhead[R]{\thepage} %Kopfzeile rechts bzw. außen


%Das hier ist neu dazugekommen für automatische Wörtertrennung.
\usepackage{xspace}



\setlength\parindent{0pt}




% zum abaendern wenn die Erklaerungen ausgeblendet werden sollen
\newcommand{\invisible}[1]{#1}


\newcommand{\lecture}[2]{\color{black} 
	
	\begin{textblock*}{3cm}(-2.4cm,0.125cm)
		{\bf Lecture {#1}}
	\end{textblock*}
	
	\begin{textblock*}{3cm}(-2.4cm,0.625cm)
		#2
	\end{textblock*}
}
\newcommand{\tutorial}[2]{\color{black} 
	
	\begin{textblock*}{3cm}(-2.4cm,0.125cm)
		{\bf Tutorial {#1}}
	\end{textblock*}
	
	\begin{textblock*}{3cm}(-2.4cm,0.625cm)
		#2
	\end{textblock*}
}



%cpp stuff

%\setmonofont{Consolas} %to be used with XeLaTeX or LuaLaTeX
\definecolor{bluekeywords}{rgb}{0,0,1}
\definecolor{greencomments}{rgb}{0,0.5,0}
\definecolor{redstrings}{rgb}{0.64,0.08,0.08}
\definecolor{xmlcomments}{rgb}{0.5,0.5,0.5}
\definecolor{types}{rgb}{0.17,0.57,0.68}

\usepackage{listings}
\lstset{language=C++,
	captionpos=b,
	%numbers=left, %Nummerierung
	%numberstyle=\tiny, % kleine Zeilennummern
	frame=lines, % Oberhalb und unterhalb des Listings ist eine Linie
	showspaces=false,
	showtabs=false,
	breaklines=true,
	showstringspaces=false,
	breakatwhitespace=true,
	escapeinside={(*@}{@*)},
	commentstyle=\color{greencomments},
	morekeywords={partial, var, value, get, set},
	keywordstyle=\color{bluekeywords},
	stringstyle=\color{redstrings},
	basicstyle=\ttfamily\small,
}


% \newcommand{\erklaerung}[1]{}
%opening
\title{Advanced Enterprise Computing - Lecturenotes SoSe2016}
\author{Julius Hülsmann}
	\pagestyle{fancy}
\begin{document}
  \maketitle
  \tableofcontents
  \newpage
\section{Replication and State Management\\ (25.04. - 09.05.)}


\subsection{Motivation and Background}

\subsubsection{Replication}
\paragraph{Definition - Replication}
Process of maintaining multiple Copies of an Entity (Data / Process / File ...)

\paragraph{Advantages of Replication in General}
\begin{itemize}
	\item {\it System Availability / Fault tolerance} in case \\
	A \quad Server fails \\
	B \quad Data is corrputed.
	\item {\it Performance / Scalability } \\
	A \quad Workloads are spread across distributed Replicas \\
	B \quad Geodistribution for processing demands in client's proximity
\end{itemize} 

\paragraph{Disadvantages of Replication in General}
\begin{itemize}
	\item Consistency vs. Performance
\end{itemize}
\paragraph{Kinds of Replication}
In general there are the following kinds of \glqq physical\grqq ~Replication. We do only consider (B).\\
\color{red} !!!!!!!!!!!!!!!!ueberpruefen am Ende!!!!!!!!!!!!!!! \color{black}
%\grafik{../src/Grafik01}{(A) Improves the availability only in case the Server Replicas use a replica cache coherence mechanism
%	(B) Improves the availability. Usually ment by the term \glqq Replication \grqq.}
\paragraph{Replication Strategies}

PAGE 17

Synchronous vs. Asynchronous
\begin{description}
	\item[Synchronous / eager] 
	\item[Asynchronous / lazy] 
\end{description}

Primary Copy vs. Update Everywhere
\begin{description}
	\item[Primary Copy / master] 
	\item[Update Everywhere / group] 
\end{description}
\paragraph{ACID}~\\

\begin{table}
	
	Name 				explanation				Protokolltitle 		implementation
	Atomacity		either perform Transaction entirely or roll-back	atomic commitement protocol	2PC
	
	
	
\end{table}

Atomiticity: \\
Consistency: does not mean Data-Consistency but that the transaction produces consistent changes.\\
Isolation: Transactions are isolated from one another\\
Durability: Once the transaction is ready (commits) it remains.

~\\
Both the Atomitcity and the Isolation are managed by the {\bf Transaction Manager}
{\it It acquires locks on behalf of all transactions and tries to come up with a serializable execution, that is, make it look like the transactions were executed one after the other.
If the transactions follow 2 Phase Locking, serializability is guaranteed. Thus, the scheduler only needs to enforce 2PL behavior.}


\paragraph{What happens to ACID in case of Replication?}
Atomicity can be guaranteed using 2PC (but expensive)
Problem: Serialization order must be the same at all replicas.

{\bf Synchronous} ACID properties apply to all copy updates

% Please add the following required packages to your document preamble:
% \usepackage{multirow}
% \usepackage[table,xcdraw]{xcolor}
% If you use beamer only pass "xcolor=table" option, i.e. \documentclass[xcolor=table]{beamer}
\begin{table}[]
	\centering
	\caption{My caption}
	\label{my-label}
	\begin{tabular}{|
			>{\columncolor[HTML]{C0C0C0}}l |l|
			>{\columncolor[HTML]{FFFFFF}}l |}
		\hline
		& \cellcolor[HTML]{C0C0C0}Procedure                                                                                                                  & \cellcolor[HTML]{C0C0C0}Advantages / Disadvantages                                                                                                                                                                                        \\ \hline
		\cellcolor[HTML]{C0C0C0}                                    &                                                                                                                                                    & {\color[HTML]{9AB299} \begin{tabular}[c]{@{}l@{}}- ACID (no \\ Inconsistencies)\end{tabular}}                                                                                                                                             \\ \cline{3-3} 
		\multirow{-2}{*}{\cellcolor[HTML]{C0C0C0}Synchronous}       & \multirow{-2}{*}{\begin{tabular}[c]{@{}l@{}}1 propagate Data \\  to everybody\\ \\ 2 Wait until \\ everybody responded\\ \\ 3 commit\end{tabular}} & {\color[HTML]{A38C8C} \begin{tabular}[c]{@{}l@{}}- High response time \\ (high execution time, \\ response time)\\ - Availability \\ (in case one Copy fails)\end{tabular}}                                                               \\ \hline
		\cellcolor[HTML]{C0C0C0}                                    &                                                                                                                                                    & {\color[HTML]{9AB299} \begin{tabular}[c]{@{}l@{}}- Response Time\\ - Availability\end{tabular}}                                                                                                                                           \\ \cline{3-3} 
		\multirow{-2}{*}{\cellcolor[HTML]{C0C0C0}Asynchronous}      & \multirow{-2}{*}{\begin{tabular}[c]{@{}l@{}}1 Update local copy\\ \\ 2 commit\\ \\ 3 Propagate Data\end{tabular}}                                  & {\color[HTML]{A38C8C} \begin{tabular}[c]{@{}l@{}}- Data inconsistency \\ (local read does not always \\ return the latest value)\\ - No guarantee that the changes \\ arrive at each copy\\ - Replication is not guaranteed\end{tabular}} \\ \hline
		\cellcolor[HTML]{C0C0C0}                                    &                                                                                                                                                    & {\color[HTML]{9AB299} }                                                                                                                                                                                                                   \\ \cline{3-3} 
		\multirow{-2}{*}{\cellcolor[HTML]{C0C0C0}Primary Copy}      & \multirow{-2}{*}{\begin{tabular}[c]{@{}l@{}}one Primary copy and\\  several read-only copies\end{tabular}}                                         & {\color[HTML]{A38C8C} }                                                                                                                                                                                                                   \\ \hline
		\cellcolor[HTML]{C0C0C0}                                    &                                                                                                                                                    & {\color[HTML]{9AB299} }                                                                                                                                                                                                                   \\ \cline{3-3} 
		\multirow{-2}{*}{\cellcolor[HTML]{C0C0C0}Update Everywhere} & \multirow{-2}{*}{\begin{tabular}[c]{@{}l@{}}Each site is able to\\  initiate changes\end{tabular}}                                                 & {\color[HTML]{A38C8C} }                                                                                                                                                                                                                   \\ \hline
	\end{tabular}
\end{table}

\subsection{Managing Replication}
\subsection{Implications of Replication}
\subsection{Paxos and CRDTs}


\section{Prototyping}
\section{Experiments}
\section{DevOps and Microservices}
\section{Reading Assignment}


\newpage
\section{Lecturenotes}
\lecture{05?}{2016-05-09} start @ 81
Für Donnerstag paper mitbringen und Paxos anschauen.

\paragraph{Paxos} (Represent as State-machine) - P. 77


Proposer


Phase 1
- Proposer choses Number largr than any value chosen before by Propposer. 
- Broadcast the integer {\it prepare(n)}, e.g. prepare(50)

Acceptors
a) Not respond at all
b) {\it recject} Reject, in case a higher value has been accepted. 50 < something
b) {\it prommise(n)} in case 50 > everything. Also Send everything that has already been accepted.

If prposer receives majority of prommise respons, -> proceed to Phase 2 ELSE -> Phase 1

Phase 2
- Check whether any <n, value> have been returned. 
- YES: take max n's value
- accept (n, value)

\paragraph{Xtensions Paxos}
	
	{\bf Multi-paxos}
	Determine Leader once
	Stay in phase 2, attatch the leader identifier
	Leader is the one to accept values
	
	{\it Purpose:  Optimize Speed} (get rid of the first phase, Master-Slave setup)
	
	
	{\bf Fast Paxos}
	
	{\bf Generalized Paxos}
	- Assumption: The execution order does not matter.

\paragraph{CRDT}
Conflict free / Communitive replicated Datatypes

Some operations are commutative, others not.

State- Based vs. Operation based.

theoretically it is possible to converge them but ... practice




IDEA
INTEGER  - example: 
e.g. not store int values but operations (increment / decrement))

SET - example

State - based Set


\section{Begriffe und Abkürzungen}

\begin{description}
	\item[Replication] Strategy to maintain mutiple copies of an entity on multiple Servers.
	\item[Replica] 
	\item[CRDT] {\it conflict-free replicated data}
	\item[Paxos] 
	\item[Commit] In case a Transaction commits, it is ready.
	\item[Concurrency control protocol] guarantees isolation of Transactions
	\item[2PL] Two phase locking (one concurrency control protocol)
	\item[Snapshot Isolation] other cuncurrency control protocol implementation
	\item[atomic commitement protocol] guarantees atomaticity
	\item[2PC] Two phase Commit
	\item[Transaction Manager] Middleware Component; Manages Atomacity and Isolation of Transactions
	\item[ACID]  Atomacity + Consistency + Isolation + Durability
\end{description}
\end{document}